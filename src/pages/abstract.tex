%*******************************************************
% Abstract
%*******************************************************
%\renewcommand{\abstractname}{Abstract}
\pdfbookmark[1]{Abstract}{Abstract}
\begingroup
\let\clearpage\relax
\let\cleardoublepage\relax
\let\cleardoublepage\relax

% \chapter*{Abstract}
% Short summary of the contents in English\dots


\vfill

\pdfbookmark[1]{Zusammenfassung}{Zusammenfassung}
\chapter*{Zusammenfassung}
Die Erstellung von modernen Unternehmensanwendungen ist ein komplexes
Themengebiet und die dabei verwendeten Technologien und Konzepte sind einem
ständigen Wandel unterzogen. In dieser Arbeit soll eine Architektur für diese Art
von Anwendungen vorgestellt werden, die diese Komplexität strukturieren und
durch verschiedene Implementierungen an einigen Stellen auch reduzieren soll.

Neben der Vorstellung einer Gesamtarchitektur sollen dabei zwei Bereiche dieser
Architektur genauer betrachtet werden. Zum einen wird das Konzept des Service
Layer beschrieben, und anhand der allgemeinen Eigenschaften von
Unternehmensanwendungen eine Reihe von generischen Service Funktionalitäten
hergeleitet und implementiert. Zum anderen soll ein zentraler
Ausführungsmechanismus entwickelt werden, der es erlaubt die Operationen des
Service Layer für den Aufruf über entfernte Methodenaufrufe in verschiedenen
Protokollen anzubieten. Anschließend werden drei dieser Protokolle und
entsprechende Implementierungen, die diesen gemeinsamen Ausführungsmechanismus
verwenden, vorgestellt.

Zum Abschluss der Arbeit werden verschiedene Anwendungsmöglichkeiten für diese
Implementierungen sowie mögliche Variationen der Architektur aufgezeigt.

\endgroup			

\vfill
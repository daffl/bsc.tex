\myChapter{Ausblick}\label{ch:final}

\section{Entwicklungs- und Testumgebung}
Die Entwicklung erfolgte mit \emph{Eclipse}\footnote{\url{http://eclipse.org}}
als Entwicklungsumgebung und \emph{Ubuntu
Linux}\footnote{\url{http://www.ubuntu.com}} 9.10 (64 bit Desktop Edition) als
Betriebssystem. Für die Verwaltung des Projekts wurde das Online
Projektmanagementsystem \emph{Redmine}\footnote{\url{http://www.redmine.org}} und
das Versionsverwaltungssystem
\emph{Subversion}\footnote{\url{http://subversion.apache.org}} auf einem
virtuellen Server in einem Frankfurter Rechenzentrum installiert. Dieses System
soll auch für die Weiterentwicklung und eventuelle Veröffentlichungen dienen. In
Eclipse selbst wurden verschiedene Plugins verwendet:
\begin{description}
  \item[Spring IDE] Spring IDE ist ein Eclipse Plugin, das die Arbeit mit dem
  Spring Framework unterstützt. Dazu gehören automatische Vervollständigung von
  Klassennamen und Eigenschaften in einer Context Definition sowie weitere
  Möglichkeiten zur Verwaltung des Context. Spring IDE wurde inzwischen in die
  Springsource Tool Suite
  \footnote{\url{http://www.springsource.com/products/sts}} integriert.
  \item[Groovy Eclipse Plugin] Für die Entwicklung in Groovy wurde das Groovy
  Eclipse Plugin\footnote{\url{http://groovy.codehaus.org/Eclipse+Plugin}}
  verwendet. Dadurch kann mit Groovy oder Java/Groovy Projekten ähnlich
  gearbeitet werden, wie schon aus der Java Entwicklung in Eclipse gewohnt.
  \item[IvyDE] Ivy bietet die Möglichkeit, verschiedene Versionen von
  Bibliotheken und deren Abhängigkeiten von öffentlichen und nichtöffentlichen
  Repository Servern\footnote{Bekanntester öffentlich verfügbarer Server ist
  \url{http://mvnrepository.com}} zu beziehen. Dafür wird in einem Eclipse
  Projekt eine \ac{XML} Konfigurationsdatei angelegt, in der alle
  Abhängigkeiten dieses Projekts definiert werden. Ivy bezieht nun diese
  Bibliotheken von den Repository Servern und macht sie in einem Projekt
  verfügbar. Ein weiterer Vorteil dieser Vorgehensweise ist, dass es
  nicht mehr notwendig ist, die verwendeten Java Bibliotheken mit dem Projekt in
  ein Versionskontrollsystem einchecken zu müssen.
  \item[Redmine Mylyn Connector] Der Redmine Mylyn
  Connector\footnote{\url{http://sourceforge.net/projects/redmin-mylyncon/}}
  verbindet die unter Eclipse verfügbare Aufgabenverwaltung Mylyn mit dem
  Projektmanagementsystem Redmine. Dadurch kann die Verwaltung von Aufgaben und
  Fehlerberichten in einem Projekt zum Großteil direkt in Eclipse vorgenommen
  werden.
  \item[Subclipse] Subclipse\footnote{\url{http://subclipse.tigris.org}} ist ein
  Eclipse Plugin für die Versionsverwaltung mit Subversion.
\end{description}

Für die Tests einzelner Implementierungen wurden umfangreiche Unit Tests unter
Verwendung von \emph{JUnit}\footnote{\url{http://www.junit.org}} erstellt. Auch
hierfür existiert eine Unterstützung des Spring Frameworks. Für Tests über ein
Netzwerk wurde auf dem lokalen System mit Hilfe der Virtualisierungslösung
\emph{Virtualbox}\footnote{\url{http://www.virtualbox.org}} ein
\emph{Debian}\footnote{\url{http://www.debian.org}} 5.0 System installiert. In
der virtuellen Maschine wurde die Verbindung zu dem in Abschnitt
\ref{subsec:xmlrpc} vorgestellten XML-RPC Protocol Handler mit einer PHP
Implementierung des XML-RPC
Protokolls\footnote{\url{http://phpxmlrpc.sourceforge.net/}} getestet. Zum Testen
des \ac{REST} Protocol Handlers aus Abschnitt \ref{subsec:rest} wurde die Java
Anwendung \emph{Rest
Client}\footnote{\url{http://code.google.com/p/rest-client/}} verwendet. Da bei
diesem Testaufbau beide Teilnehmer auf dem gleichen physikalischen System
arbeiteten, waren keine aussagekräftigen Lasttests möglich.

\section{Erweiterungsm\"oglichkeiten} 
Auch wenn die Ziele dieser Arbeit eine Grundlage für einen Service Layer zu
erstellen und durch die Service Engine einen protokollunabhängigen
Ausführungsmechanismus für entfernte Methodenaufrufe zu implementieren, erreicht
wurden, lässt der modulare Aufbau des Gesamtsystems viele
Erweiterungsmöglichkeiten zu.

Durch die Referenzimplementierungen der Protocol Handler für \ac{RMI}, XML-RPC
und \ac{REST} werden zwar bereits einige häufig verwendete Protokolle
unterstützt, aber vor allem in diesem Bereich besteht einiges Potential für
Erweiterungen. Nachdem in der Service Engine die Darstellung von
zustandsgebundenen und zustandslosen synchronen \ac{RPC} Aufrufen abstrahiert
wurde, wäre der nächste Schritt eine allgemeine Unterstützung für asynchrone
entfernte Methodenaufrufe zu implementieren. Dadurch ist dann beispielsweise
die Unterstützung des Webservice Protokolls SOAP in der Version 1.2 möglich.

Auch die Ausarbeitung von Details in den vorhandenen Implementierungen ist an
manchen Stellen möglich. So ist es geplant die in Kapitel \ref{chap:servicelayer}
vorgestellten Service Implementierungen zu einem Framework zu erweitern, das den
Aufbau des Service Layer einer Anwendung vorgibt. Dazu gehört auch eine bessere
Integration der Services untereinander, da diese momentan noch manuell in der
Spring Context Definition vorgenommen werden muss. Ein weiterer Punkt ist die
Optimierung der \ac{HTTP} Umsetzung (entsprechend der Spezifikation in
\cite{rfc2616}) des \ac{REST} Protokoll Handler und die Fertigstellung der
Presentation Implementierung für \ac{HTML}.

\section{Fazit}
Die Erstellung von modernen Unternehmens- und Webanwendungen ist ein komplexes
Themengebiet, das einem ständigen Wandel unterzogen ist. Diese Arbeit und die
vorgestellten Implementierungen sollen dazu beitragen diese Komplexität
an einigen Stellen etwas zu reduzieren und zu strukturieren.

Dafür wurden im Einleitungskapitel die Definition und die Herausforderungen von
modernen Unternehmensanwendungen herausgearbeitet und die Aufgabenstellung dieser
Arbeit sowie die Herangehensweise anhand der verwendeten Schichtenarchitektur
vorgestellt. 

Aufgabe war es zum einen das Konzept des Service Layer zu evaluieren
und anhand der Eigenschaften von Unternehmensanwendungen eine Reihe von Services
zu implementieren, die einen Teil dieser Aufgabe übernehmen. Diese
Implementierungen können somit in verschiedenen Anwendungen die Grundlage für
einen Service Layer bilden. Zum anderen sollte ein zentraler
Ausführungsmechanismus entwickelt werden, der die Grundlage dafür bietet, die
Methoden dieses Service Layer über entfernte Methodenaufrufe zur Verfügung zu
stellen. Darauf aufbauend sollten anschließend Reihe von \ac{RPC} Protokollen für
entfernte Methodenaufrufe evaluiert werden und die jeweiligen Implementierungen,
die diesen Ausführungsmechanismus verwenden, realisiert werden.

Ein Grundlagenkapitel erstellte einen Überblick über die verwendeten
Technologien, Konzepte und Begriffe, die für das weitere Verständnis wichtig
waren. Ein Schwerpunkt lag dabei auf der Java Plattform und den dort möglichen
Programmiersprachen sowie auf verschiedenen Java Technologien, unter anderem dem
Spring Framework. Weiterhin wurde das Konzept des entfernten Methodenaufrufs
(\ac{RPC}) sowie verschiedene Möglichkeiten für den Datenaustausch behandelt.

Im weiteren Verlauf wurde der Datasource Layer, der dazu dient die Verbindung zu
anderen Anwendungen herzustellen, als unterste Schicht der Architektur
vorgestellt. Darauf aufbauend wurden die verschiedenen Komponenten des Domain
Layer näher betrachtet, die bereits einen Teil der Anwendungslogik übernehmen.
Für die weitere Implementierung von Anwendungslogik wurde anschließend das
Konzept des Service Layer eingeführt, der eine gemeinsame Schnittstelle einer
Anwendung an einer zentralen Stelle definiert. In diesem Zusammenhang wurden
dann einige Service Implementierungen vorgestellt, die aufgrund der allgemeinen
Anforderungen von Unternehmensanwendungen in verschiedenen
Anwendungen zum Einsatz kommen können.

Das nachfolgende Kapitel beschäftigte sich mit dem Remoting Layer, der dazu dient
den Service Layer einer Anwendung über verschiedene \ac{RPC} Protokolle
anzubinden. Hier wurde die Implementierung der Service Engine vorgestellt, die
die Aufgabe übernimmt die abstrahierte Form eines Methodenaufrufs zu verarbeiten
und auf einem entsprechenden Service auszuführen. Durch diese allgemeine Form
eines Methodenaufrufs ist es nun möglich verschiedene \ac{RPC} Protokolle über
den zentralen Ausführungsmechanismus der Service Engine anzubinden. Dazu wurden
in den folgenden Abschnitten Referenzimplementierungen für \ac{RMI}, XML-RPC und
\ac{REST} behandelt. Weiterhin wurden dann im Rahmen des Gesamtkonzeptes
verschiedene Anwendungsfälle gezeigt, wie diese Implementierungen genutzt werden
können.

Zusammenfassend lässt sich sagen, dass beide Teile der Aufgabenstellung
erfolgreich umgesetzt wurden. Zum einen wurde eine Reihe von Services
hergeleitet und implementiert, die verschiedene häufige Anforderungen bei der Entwicklung
einer Unternehmensanwendung in einer generischen Form umsetzen und deshalb leicht
wiederverwendbar sind. Der Vorteil dieser Implementierungen und der Verwendung
eines Service Layer ist, dass sie unabhängig von der Präsentationsschicht einer
Anwendung arbeiten. Dies steht im Unterschied zu einem Großteil der bisher
verfügbaren Frameworks, da diese in den meisten Fällen mit einer konkreten
Präsentationsschicht arbeiten\footnote{Beispielsweise Webframeworks, GUI
Frameworks etc.}.

Zum anderen wurde eine Anwendungsschicht in Form der Service Engine entwickelt,
die eine zentrale Ausführungsumgebung für Anfragen von verschiedenen \ac{RPC}
Protokollen auf den Services des Service Layer bietet. Die Service Engine
verwendet dabei eine allgemeine Darstellung eines \ac{RPC} und implementiert
selbst kein konkretes Protokoll. Diese allgemeine Darstellung wird dann von der
übergeordneten Schicht in Form verschiedener Protocol Handler, die ein konkretes
\ac{RPC} Protokoll implementieren, verwendet. An den Referenzimplementierungen für
\ac{RMI}, XML-RPC und \ac{REST} wurde nun gezeigt, dass auch Protokolle mit
unterschiedlichen Eigenschaften und Voraussetzungen an die Service Engine
angebunden werden können. Auch hier existierten bisher nur Bibliotheken,
die ein konkretes \ac{RPC} Protokoll implementieren und auch direkt die Aufgabe
der Ausführung übernehmen, was die Unterstützung mehrerer \ac{RPC}
Protokolle deutlich komplizierter gestaltete.

\myChapter{Codebeispiele}\label{chap:appendix}

\section{Spring Framework}\label{app:spring}
Die folgenden Codebeispiele sollen die Verwendung des Dependency Injection
Container des Spring Frameworks zeigen. Zu diesem Zweck soll das bisher
verwendete User JavaBean als Grundlage verwendet werden:

\lstinputlisting[caption=User Domain Objekt (User.java), label=lst:appjavabean]
{sources/appendix/javabeanexample.java}

Anschließend wird ein Service Interface für dieses User Domain Objekt erstellt,
das die einfache Authentifizierung eines Benutzers anhang von Benutzername und
Passwort ermöglicht.

\lstset{language=Java}
\lstinputlisting[caption=Beispiel eines Service Interface (UserService.java),
label=lst:a_userservice] {sources/appendix/UserService.java}

Dieses Interface wird nun in einer sehr einfachen Form implementiert. Die
Implementierung erwartet eine Liste von Benutzer Objekten und überprüft bei
einem Aufruf der login() Methode, ob die übergebenen Authentifizierungsdaten
mit einem User Objekt übereinstimmen.

\lstinputlisting[caption=Einfache Implementierung des User Service Interface
(SimpleUserService.java)] {sources/appendix/SimpleUserService.java}

Nun muss eine Spring Context Definition erstellt werden, die die
SimpleUserService Implementierung instanziiert und dieser eine Reihe von User
Objekten übergibt.

\lstset{language=XML}
\lstinputlisting[caption=Spring Context Definition für den SimpleUSerService
(applicationContext.xml)] {sources/appendix/applicationContext.xml}

Wird der dieser Context nun in einem Hauptprogramm in den Dependency Injection
Container gelesen können die dort definierten Objekte anhand der Interfaces
benutzt werden, die diese implementieren. Das folgende Listing
\ref{lst:bootstrap} zeigt die Verwendung in einem einfachen Java
Kommandozeilenprogramm.

\lstset{language=Java}
\lstinputlisting[caption=Verwendung des Dependency Injection Containers
(Bootstrap.java), label=lst:bootstrap] {sources/appendix/Bootstrap.java}

\pagebreak
\section{XML-RPC}
Das in Abschnit \ref{app:spring} erstellte UserService Interface wird hier auch
verwendet um einen Aufruf über das XML-RPC Protokoll zu illustrieren. Ein
Methodenaufruf auf dieses Interface hat die folgende Form:

\lstset{language=XML}
\lstinputlisting[caption=XML-RPC Methodenaufruf]{sources/appendix/xmlrpcreq.txt}

Wurde der richtige Benutzer gefunden wird ein Struct Objekt, das den
entsprechenden Benutzer repräsentiert, zurückgeliefert.

\lstinputlisting[caption=Antwort auf einen
XML-RPC Methodenaufruf]{sources/appendix/xmlrpcresp.txt}

Wurde der entsprechende Benutzer nicht gefunden wird ein Fehlerobjekt in der
folgenden Form übertragen.

\lstinputlisting[caption=XML-RPC Antwort bei
einem Fehler]{sources/appendix/xmlrpcfault.txt}
